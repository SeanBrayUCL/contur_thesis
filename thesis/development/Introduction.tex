\chapter{Introduction}
\label{chapterlabel1}

Good software development does not just entail writing functional code. Factors that don't impact the final output of the code, like clear well written syntax, inclusion of code testing, and making efficient use of CPU resources for the task at hand can significantly increase the usability and reliableness of software in a research setting. This thesis focuses on making efficient use of CPU resources, or more intuitively writing code that does the required task as fast as is possible with the tools at hand. The project focuses in particular on the \textbf{Contur} package, which is open source software used in particle physics to test the consistency of theoretical predictions from newly proposed physics models against realised experimental data. The project entailed profiling the \textbf{Contur}  code base and from the profile results identifying parts of the code where potential inefficiencies exist which can be improved upon. A brief summary of the contents of each chapter of this thesis follows below.

To start the thesis in Chapter 2 we attempt to provide the necessary \textbf{Contur} background. This background contains a brief sketch of the current situation within particle physics research that creates a gap for a tool like \textbf{Contur} to be useful, followed by an overview of the \textbf{Contur}  package itself in terms of what it does and how it works. The outline of the workings of \textbf{Contur}  will be necessarily brief but should be sufficient to give the reader the required background to understand later changes made to better optimise the code.

From Chapter 3 on in this thesis we start to discuss work actually carried out by this author as part of the research project. Chapter 3 specifically begins with a discussion of the benefits to \textbf{Contur} users arising from the improved runtime which will result from code efficiencies. The chapter then moves onto outlining the tools used to perform and visualise the profiling results. The chapter then concludes by presenting the profiling results for the last version of \textbf{Contur} that exists before we started this project. This initial profile is taken as the bench mark to judge the effectiveness of later attempts to optimise \textbf{Contur}.

In Chapter 4 we digress slightly from the main theme of the project to discuss introducing automatic tests into the \textbf{Contur} package. When making code changes in \textbf{Contur} for optimisation purposes, an important consideration is that we don't unintentional break or introduce errors into \textbf{Contur} . Especially errors that will have an impact on the final output of \textbf{Contur}. One means we can reduce the chances of introducing errors is via robust testing infrastructure. Chapter 4 outlines the step taken as part of this research project to improve \textbf{Contur's} testing infrastructure.

Chapter 5 then finally outlines the changes made to \textbf{Contur}  as part of this research project. Each of the changes outlined has been made because they can carry out the exact same process as the existing \textbf{Contur}  code in a more efficient way. For each change made we give the background necessary to understand the purpose of the section of code we are changing, we explain the change we made and we give updated profiling results showing the impact of the change in terms of run time.

We will then conclude the thesis in Chapter 6 by summarising the accomplishments of the project. Additionally the final profile results of \textbf{Contur}  after all the optimisation changes have been implemented will be presented. From these final results we can understand the most computational intensive remaining parts of a \textbf{Contur}  run thus providing the target areas for future optimisation attempts with \textbf{Contur}.  

